\documentclass[a4paper,12pt]{article}
\usepackage[utf8]{inputenc}
\usepackage{amsmath, amssymb, amsthm}
\usepackage{geometry}
\usepackage{enumitem}
\usepackage{CJKutf8}
\usepackage{xcolor}
\usepackage{fancyhdr}
\usepackage{booktabs}

% 页面设置
\geometry{left=2.5cm, right=2.5cm, top=2.5cm, bottom=2.5cm}
\setlength{\parindent}{0pt}
\setlength{\parskip}{0.8em}

% 页眉页脚
\pagestyle{fancy}
\fancyhf{}
\lhead{COMP1046-E1 Exam Solutions}
\rhead{Autumn 2024/2025}
\cfoot{\thepage}

% 自定义环境
\newenvironment{solution}{\par\noindent\textbf{\color{blue}解析/Solution:}\par\nopagebreak}{\par\vspace{1em}}
\newcommand{\code}[1]{\texttt{#1}}

\begin{document}
\begin{CJK*}{UTF8}{gbsn}

\title{\textbf{COMP1046 Mathematics for Computer Scientists}\\ \large 2024/2025 Autumn Exam Paper Analysis \& Solutions}
\author{Generated by GitHub Copilot}
\date{\today}
\maketitle

\tableofcontents
\newpage

\section*{试卷概览 (Overview)}
\begin{itemize}
    \item \textbf{总分}: 100分 (4道大题, 每题25分)
    \item \textbf{难度}: 中等偏上 (需要扎实的计算功底和证明写作能力)
    \item \textbf{核心考点}:
    \begin{itemize}
        \item Q1: 逻辑、真值表、谓词翻译、数学归纳法。
        \item Q2: 集合运算、函数可逆性、等价关系证明、排列组合、基础概率。
        \item Q3: 线性方程组(秩, Rouché-Capelli, Cramer)、向量空间判定。
        \item Q4: 线性映射(核, 像, 双射)、逆像证明、矩阵表示。
    \end{itemize}
\end{itemize}

\hrule
\vspace{1cm}

\section{Question 1: Logic \& Proofs (25 marks)}

\subsection*{(a) Definitions (3 marks)}
\textbf{问题}: 定义 Proposition, Tautology, Contradiction。
\begin{solution}
\begin{enumerate}
    \item \textbf{Proposition (命题)}: 一个非真即假的陈述句 (A declarative sentence that is either true or false, but not both)。
    \item \textbf{Tautology (重言式)}: 此复合命题在所有变量取值组合下永远为真 (Always true)。
    \item \textbf{Contradiction (矛盾式)}: 此复合命题在所有变量取值组合下永远为假 (Always false)。
\end{enumerate}
\end{solution}

\subsection*{(b) Truth Tables (5 marks)}
\textbf{问题}: 画真值表并找反例,使得 $(p \to q) \land (q \to r) \land (r \to p)$ 为真,但 $p \land q \land r$ 为假。
\begin{solution}
目标是让前者为 True (即 $p \to q, q \to r, r \to p$ 三者形成循环蕴含,全真),后者为 False。
\begin{itemize}
    \item 赋值: \textbf{$p=F, q=F, r=F$}。
    \item 验证前者: $F \to F$ 是 True。三个都是 True,且运算结果为 True。
    \item 验证后者: $F \land F \land F$ 是 False。
\end{itemize}
\textbf{答案}: $p=False, q=False, r=False$.
\end{solution}

\subsection*{(c) Predicate Logic (12 marks)}
\textbf{问题}: 翻译与推理。$P(x,y)$: $x$边数多于$y$; $Q(x)$: 方形; $R(x)$: 六边形。
\begin{solution}
\begin{enumerate}
    \item[(i)] \textbf{翻译}:
    \begin{itemize}
        \item (A) No square is a hexagon: $\neg \exists x (Q(x) \land R(x))$ 或 $\forall x (Q(x) \to \neg R(x))$。
        \item (B) All hexagons have more sides than any square: $\forall x (R(x) \to \forall y (Q(y) \to P(x, y)))$。
    \end{itemize}
    \item[(ii)] \textbf{翻译回英文}:
    \begin{itemize}
        \item (C) $\exists x(Q(x) \land \forall y(P(y, x) \to \neg Q(y)))$:
        \item "There exists a square such that every polygon that has more sides than it must not be a square." (存在一个正方形,比它边数多的多边形都不是正方形)。
    \end{itemize}
    \item[(iii)] \textbf{Inference}:
    \begin{itemize}
        \item Can (A) be inferred from (B) and (C)? \textbf{No (不能)}.
        \item 虽然几何常识告诉我们正方形不是六边形,但仅从逻辑语句(B)和(C)无法推导出(A)。
        \item \textbf{反例}: 假设宇宙中只有一个元素 $a$,它既是方形 ($Q(a)$) 又是六边形 ($R(a)$)。
        \item 这种情况下 (A) 为假。但只要 $P(a, a)$ 为假,(B) 和 (C) 均可能在特定模型下成立或不冲突(取决于 $P$ 的定义)。
    \end{itemize}
\end{enumerate}
\end{solution}

\subsection*{(d) Induction (5 marks)}
\textbf{问题}: 证明 $5 \mid (6^n - 1)$ 对 $n \in \mathbb{Z}^+$ 成立。
\begin{solution}
\begin{enumerate}
    \item \textbf{Base Case ($n=1$)}: $6^1 - 1 = 5$,$5 \mid 5$ 成立。
    \item \textbf{Hypothesis}: 假设 $n=k$ 时成立,即 $6^k - 1 = 5m$ ($m \in \mathbb{Z}$),implies $6^k = 5m + 1$。
    \item \textbf{Inductive Step}: 考虑 $n=k+1$:
    \[ 6^{k+1} - 1 = 6 \cdot 6^k - 1 \]
    代入假设:
    \[ = 6(5m + 1) - 1 = 30m + 6 - 1 = 30m + 5 = 5(6m + 1) \]
    因为 $6m+1$ 是整数,所以 $5 \mid (6^{k+1}-1)$。
    \item \textbf{Conclusion}: By PMI, true for all $n \ge 1$。
\end{enumerate}
\end{solution}

\newpage
\section{Question 2: Discrete Math (25 marks)}

\subsection*{(a) Sets (5 marks)}
$X=\{5,8,12,15\}, Y=\{1,8,12,13,18\}$.
\begin{solution}
\begin{itemize}
    \item (i) $|Y| - |X| = 5 - 4 = 1$.
    \item (ii) $X \cup Y = \{1, 5, 8, 12, 13, 15, 18\}$.
    \item (iii) $Y - X = \{1, 13, 18\}$ (在Y但不在X).
    \item (iv) $\mathcal{P}(X \cap Y)$: $X \cap Y = \{8, 12\}$。幂集为 $\{\emptyset, \{8\}, \{12\}, \{8, 12\}\}$.
\end{itemize}
\end{solution}

\subsection*{(b) Function (2 marks)}
$f(x) = (x-1)(x-2)$. Is it invertible?
\begin{solution}
\textbf{No}. 
因为 $f(1)=0$ 且 $f(2)=0$。
$1 \neq 2$ 但映射到同一个值,所以不单射 (Not injective),故不可逆。
\end{solution}

\subsection*{(c) Equivalence Relation (6 marks)}
 Relation on $X \times X$: $(a,b)R(c,d) \iff ad = bc$.
\begin{solution}
\begin{itemize}
    \item \textbf{Reflexive}: $(a,b)R(a,b) \iff ab = ba$. 乘法交换律,成立。
    \item \textbf{Symmetric}: 若 $(a,b)R(c,d) \implies ad=bc$. 则 $cb=da \implies (c,d)R(a,b)$. 成立。
    \item \textbf{Transitive}: 若 $ad=bc$ 且 $cf=de$.
    \[ \frac{a}{b} = \frac{c}{d} \quad \text{and} \quad \frac{c}{d} = \frac{e}{f} \implies \frac{a}{b} = \frac{e}{f} \implies af=be \]
    成立 (注: 严谨写法可用乘法式子推导 $adf = bcf, bcf = bde \implies adf = bde \implies af = be$ 假设非零)。
\end{itemize}
\end{solution}

\subsection*{(d) Counting (5 marks)}
Digits: $\{0,1,2,3,4,5,6\}$ (Total 7). No repetition.
\begin{solution}
\begin{enumerate}
    \item[(i)] \textbf{7-digit odd}:
    \begin{itemize}
        \item 末位(奇数): $1,3,5$ (3 种选择)。
        \item 首位(非0): 除去末位,剩6个数(含0)。0不能在首位,所以首位有 5 种选择。
        \item 中间5位: 剩下5个数全排列 $P(5,5) = 120$。
        \item 总数: $3 \times 5 \times 120 = 1800$。
    \end{itemize}
    \item[(ii)] \textbf{4-digit divisible by 25}:
    \begin{itemize}
        \item 末两位只能是: 25, 50 (因为没有75, 00重复不可用)。
        \item Case **XX25**: 剩 $\{0,1,3,4,6\}$. 首位非0 (4种), 次位 (4种) $\to 16$。
        \item Case **XX50**: 剩 $\{1,2,3,4,6\}$. 首位 (5种), 次位 (4种) $\to 20$。
        \item 总数: $16 + 20 = 36$.
    \end{itemize}
\end{enumerate}
\end{solution}

\subsection*{(e) Probability (7 marks)}
\begin{solution}
\begin{enumerate}
    \item[i.] Pay 9, Win $2n$. Need $2n > 9 \implies n > 4.5$.
    Dice must be 5 or 6. Prob = $2/6 = 1/3$.
    \item[ii.] Pay 11, Win $2 \times \max(d_1, d_2)$. Need $2\max > 11 \implies \max = 6$.
    Outcomes with max 6:
    $(6,1)\dots(6,5)$ (5个), $(1,6)\dots(5,6)$ (5个), $(6,6)$ (1个).
    Total 11 favorable outcomes. Sample space 36.
    Prob = $11/36$.
\end{enumerate}
\end{solution}

\newpage
\section{Question 3: Linear Systems (25 marks)}

\subsection*{(a) System Solving}
\begin{solution}
\begin{enumerate}
    \item[(i)] \textbf{Complete Matrix $A_c$}:
    \[ \begin{pmatrix} 3 & 2 & 0 & 1 & | & 7 \\ 0 & 2 & -1 & 1 & | & 1 \\ 1 & -1 & 3 & 0 & | & 1 \\ 0 & 4 & -2 & 2 & | & 2 \end{pmatrix} \]
    \item[(ii)] \textbf{Rank}:
    Observe Row 4 is $2 \times$ Row 2. So Row 4 is redundant.
    Rank of $A$ (first 4 cols) is 3. Rank of $A_c$ is 3.
    \item[(iii)] \textbf{Rouché-Capelli}:
    $r(A) = r(A_c) = 3 < n=4$.
    System is \textbf{Compatible Undetermined} (Infinite solutions, 1 free parameter).
    \item[(iv)] \textbf{Cramer's Rule} (Given $x_3=0$):
    System reduces to:
    $3x_1 + 2x_2 + x_4 = 7$ \\
    $2x_2 + x_4 = 1$ \\
    $x_1 - x_2 = 1$
    
    Determinant $D = \det \begin{pmatrix} 3 & 2 & 1 \\ 0 & 2 & 1 \\ 1 & -1 & 0 \end{pmatrix} = 3$.
    Determinant $D_{x2} = \det \begin{pmatrix} 3 & 7 & 1 \\ 0 & 1 & 1 \\ 1 & 1 & 0 \end{pmatrix} = 3$.
    $x_2 = 3/3 = 1$.
\end{enumerate}
\end{solution}

\subsection*{(b) Vector Spaces $E, F, G, H$}
\begin{solution}
$E$: Plane (Subspace). $F$: Inequality (Not subspace). $G$: Line in 3D (Subspace). $H$: Line in 2D.
\begin{itemize}
    \item (i) $E$ closed. $F$ not closed (e.g., negative scalar).
    \item (ii) True/False:
    \begin{itemize}
        \item S1, S3, S4: \textbf{True}. (E, G, H 都是各自空间的向量空间)。
        \item S2: \textbf{False}. (F not a vector space).
        \item S5: \textbf{True}. ($=$ implies $\ge$).
        \item S6: \textbf{False}. (Subspace 必须是 Vector Space 的子集,F 连 Space 都不是)。
        \item S7: \textbf{True}. ($G \subset E$ 且 G 是由 E 的特例 $z=0$ 切出来的)。
        \item S8: \textbf{False}. ($H$ 是 $\mathbb{R}^2$, $E$ 是 $\mathbb{R}^3$, 维度不容)。
    \end{itemize}
\end{itemize}
\end{solution}

\newpage
\section{Question 4: Linear Mappings (25 marks)}

$f(x,y,z) = (3z, 2y+z, x-y-z)$, $g$ is inverse.

\begin{solution}
\begin{enumerate}
    \item[(a)] $f(3,2,1) = (3(1), 2(2)+1, 3-2-1) = (3, 5, 0)$.
    \item[(b)] \textbf{Endomorphism? Yes}. Domain = Codomain = $\mathbb{R}^3$.
    \item[(c)] \textbf{Bijective?}
    Matrix $A = \begin{pmatrix} 0 & 0 & 3 \\ 0 & 2 & 1 \\ 1 & -1 & -1 \end{pmatrix}$.
    $\det(A) = 3(0 - 2) = -6 \neq 0$. Invertible matrix $\implies$ Bijective map.
    \item[(d)] \textbf{Ker(f)}: Since bijective, $\ker(f) = \{\mathbf{0}\}$.
    \item[(e)] \textbf{Dim(Ker)}: 0.
    \item[(f)] \textbf{Rank-Nullity}: $3 = 0 + \dim(\text{Im})$. Image dim is 3.
    \item[(g)] \textbf{Proof $g(U) = f^{-1}(U)$}:
    \begin{itemize}
        \item Let $y \in g(U) \implies y = g(x)$ for some $x \in U$. Apply $f$: $f(y) = f(g(x)) = x \in U$. So $y \in f^{-1}(U)$.
        \item Inverse direction similar.
    \end{itemize}
    \item[(h)] \textbf{Matrix A}: See part (c).
    \item[(i)] \textbf{Matrix B} ($A^{-1}$):
    Calculation of inverse:
    \[ B = \frac{1}{-6} \begin{pmatrix} -1 & -3 & -6 \\ 1 & -3 & 0 \\ -2 & 0 & 0 \end{pmatrix} \]
    \item[(j)] \textbf{Product BA}: Identity Matrix $I$.
\end{enumerate}
\end{solution}

\end{CJK*}
\end{document}