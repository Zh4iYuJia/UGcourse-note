\documentclass[a4paper,12pt]{article}
\usepackage[utf8]{inputenc}
\usepackage{geometry}
\usepackage{longtable}
\usepackage{booktabs}
\usepackage{CJKutf8}
\usepackage{amsmath, amssymb}
\usepackage{xcolor}

% 页面设置
\geometry{left=2cm, right=2cm, top=2cm, bottom=2cm}

% 表格样式
\renewcommand{\arraystretch}{1.2}

\begin{document}
\begin{CJK*}{UTF8}{gbsn}

\title{\textbf{COMP1046 考前必备英文术语表 (Exam Vocabulary)}}
\author{}
\date{}
\maketitle

这份词汇表涵盖了离散数学和线性代数的核心术语,按主题分类,帮助你确保读题无障碍。

\section*{1. 逻辑与证明 (Logic \& Proofs)}
\begin{longtable}{p{4cm} p{4cm} p{7cm}}
\toprule
\textbf{English Term} & \textbf{Chinese Meaning} & \textbf{Notes} \\
\midrule
\textbf{Proposition} & 命题 & 非真即假的陈述句 \\
\textbf{Tautology} & 重言式 / 永真式 & 永远为 True \\
\textbf{Contradiction} & 矛盾式 / 永假式 & 永远为 False \\
\textbf{Contingency} & 可能式 & 既非永真也非永假 \\
\textbf{Implication} & 蕴含 ($p \to q$) & If p then q \\
\textbf{Converse} & 逆命题 & $q \to p$ \\
\textbf{Inverse} & 否命题 & $\neg p \to \neg q$ \\
\textbf{Contrapositive} & 逆否命题 & $\neg q \to \neg p$ (与原命题同真假) \\
\textbf{Predicate} & 谓词 & 如 $P(x)$ \\
\textbf{Domain} & 论域 / 定义域 & 变量 $x$ 的取值范围 \\
\textbf{Quantifier} & 量词 & $\forall$ (Universal), $\exists$ (Existential) \\
\textbf{Induction} & 归纳法 & Base case, Hypothesis, Inductive step \\
\bottomrule
\end{longtable}

\section*{2. 集合与函数 (Sets \& Functions)}
\begin{longtable}{p{4cm} p{4cm} p{7cm}}
\toprule
\textbf{English Term} & \textbf{Chinese Meaning} & \textbf{Notes} \\
\midrule
\textbf{Set} & 集合 & \\
\textbf{Subset} & 子集 & $\subseteq$ \\
\textbf{Power Set} & 幂集 & $\mathcal{P}(A)$, 所有子集的集合 \\
\textbf{Cardinality} & 势 / 基数 & $|A|$, 元素个数 \\
\textbf{Intersection} & 交集 & $\cap$ \\
\textbf{Union} & 并集 & $\cup$ \\
\textbf{Difference} & 差集 & $A - B$ or $A \setminus B$ \\
\textbf{Function / Map} & 函数 / 映射 & $f: A \to B$ \\
\textbf{Domain} & 定义域 & 输入集合 A \\
\textbf{Codomain} & 陪域 & 可能的输出集合 B \\
\textbf{Image / Range} & 像 / 值域 & 实际的输出集合 $f(A)$ \\
\textbf{Injective} & 单射 & One-to-one (不重叠) \\
\textbf{Surjective} & 满射 & Onto (覆盖B) \\
\textbf{Bijective} & 双射 & One-to-one correspondence \\
\textbf{Invertible} & 可逆 & 存在反函数 (必须是 Bijective) \\
\textbf{Composition} & 复合 & $f \circ g$ \\
\bottomrule
\end{longtable}

\section*{3. 关系 (Relations)}
\begin{longtable}{p{4cm} p{4cm} p{7cm}}
\toprule
\textbf{English Term} & \textbf{Chinese Meaning} & \textbf{Notes} \\
\midrule
\textbf{Relation} & 关系 & $a R b$ \\
\textbf{Reflexive} & 自反性 & $\forall x, xRx$ \\
\textbf{Symmetric} & 对称性 & $xRy \to yRx$ \\
\textbf{Anti-symmetric} & 反对称性 & $xRy \land yRx \to x=y$ \\
\textbf{Transitive} & 传递性 & $xRy \land yRz \to xRz$ \\
\textbf{Equivalence Relation} & 等价关系 & Ref + Sym + Trans \\
\textbf{Partial Order} & 偏序关系 & Ref + Anti-sym + Trans \\
\bottomrule
\end{longtable}

\newpage

\section*{4. 线性代数基础 (Linear Algebra Basics)}
\begin{longtable}{p{4cm} p{4cm} p{7cm}}
\toprule
\textbf{English Term} & \textbf{Chinese Meaning} & \textbf{Notes} \\
\midrule
\textbf{Matrix} & 矩阵 & Plural: Matrices \\
\textbf{Square Matrix} & 方阵 & 行数=列数 \\
\textbf{Identity Matrix} & 单位矩阵 & $I$ \\
\textbf{Determinant} & 行列式 & $\det(A)$ \\
\textbf{Trace} & 迹 & $\text{tr}(A)$, 对角线之和 \\
\textbf{Transpose} & 转置 & $A^T$ \\
\textbf{Singular} & 奇异的 & 不可逆 ($\det=0$) \\
\textbf{Non-singular} & 非奇异的 & 可逆 ($\det \neq 0$) \\
\textbf{Invertible} & 可逆的 & 存在 $A^{-1}$ \\
\textbf{Rank} & 秩 & 线性无关行/列的最大数目 \\
\bottomrule
\end{longtable}

\section*{5. 线性方程组 (Systems of Linear Equations)}
\begin{longtable}{p{4cm} p{4cm} p{7cm}}
\toprule
\textbf{English Term} & \textbf{Chinese Meaning} & \textbf{Notes} \\
\midrule
\textbf{System} & 方程组 & \\
\textbf{Augmented Matrix} & 增广矩阵 & $(A \mid b)$ \\
\textbf{Homogeneous} & 齐次的 & 等号右边全是 0 ($Ax=0$) \\
\textbf{Consistent} & 有解的 & 也叫 Compatible \\
\textbf{Inconsistent} & 无解的 & 也叫 Incompatible \\
\textbf{Determined} & 唯一解 & Rank = 变量数 \\
\textbf{Undetermined} & 无穷多解 & Rank < 变量数 \\
\textbf{Row Echelon Form} & 行阶梯形 & 用于求秩 \\
\bottomrule
\end{longtable}

\section*{6. 向量空间 (Vector Spaces)}
\begin{longtable}{p{4cm} p{4cm} p{7cm}}
\toprule
\textbf{English Term} & \textbf{Chinese Meaning} & \textbf{Notes} \\
\midrule
\textbf{Vector Space} & 向量空间 & \\
\textbf{Subspace} & 子空间 & 包含0, 加法封闭, 数乘封闭 \\
\textbf{Linear Combination} & 线性组合 & $c_1 v_1 + \dots + c_n v_n$ \\
\textbf{Linearly Independent} & 线性无关 & \\
\textbf{Linearly Dependent} & 线性相关 & \\
\textbf{Span} & 张成 & 生成的空间 \\
\textbf{Basis} & 基 & 线性无关且张成全空间的一组向量 \\
\textbf{Dimension} & 维数 & 基向量的个数 \\
\bottomrule
\end{longtable}

\section*{7. 线性映射 (Linear Mappings)}
\begin{longtable}{p{4cm} p{4cm} p{7cm}}
\toprule
\textbf{English Term} & \textbf{Chinese Meaning} & \textbf{Notes} \\
\midrule
\textbf{Mapping / Map} & 映射 & \\
\textbf{Linear Mapping} & 线性映射 & $f(u+v)=f(u)+f(v)$ \\
\textbf{Affine Mapping} & 仿射映射 & $f(x)=Ax+b$ \\
\textbf{Endomorphism} & 自同态 & $V \to V$ \\
\textbf{Kernel} & 核 & $\ker(f) = \{v \mid f(v)=0\}$ \\
\textbf{Image} & 像 & $\text{Im}(f)$ \\
\textbf{Rank-Nullity Thm} & 秩-零化度定理 & $\dim(V) = \dim(\ker) + \dim(\text{Im})$ \\
\bottomrule
\end{longtable}

\end{CJK*}
\end{document}